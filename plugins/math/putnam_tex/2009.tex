\documentclass[amssymb,twocolumn,pra,10pt,aps]{revtex4-1}
\usepackage{mathptmx,amsmath}

\begin{document}
\title{The 70th William Lowell Putnam Mathematical Competition \\
    Saturday, December 5, 2009}
\maketitle

\begin{itemize}

\item[A--1]
Let $f$ be a real-valued function on the plane such that for every
square $ABCD$ in the plane, $f(A)+f(B)+f(C)+f(D)=0$. Does it follow that
$f(P)=0$ for all points $P$ in the plane?

\item[A--2]
Functions $f,g,h$ are differentiable on some open interval around $0$
and satisfy the equations and initial conditions
\begin{gather*}
f' = 2f^2gh+\frac{1}{gh},\quad f(0)=1, \\
g'=fg^2h+\frac{4}{fh}, \quad g(0)=1, \\
h'=3fgh^2+\frac{1}{fg}, \quad h(0)=1.
\end{gather*}
Find an explicit formula for $f(x)$, valid in some open interval around $0$.

\item[A--3]
Let $d_n$ be the determinant of the $n \times n$ matrix whose entries, from
left to right and then from top to bottom, are $\cos 1, \cos 2, \dots, \cos
n^2$. (For example,
\[
 d_3 = \left| \begin{matrix} \cos 1 & \cos 2 & \cos 3 \\
               \cos 4 & \cos 5 & \cos 6 \\
\cos 7 & \cos 8 & \cos 9
              \end{matrix} \right|.
\]
The argument of $\cos$ is always in radians, not degrees.) Evaluate
$\lim_{n\to\infty} d_n$.

\item[A--4]
Let $S$ be a set of rational numbers such that
\begin{enumerate}
\item[(a)] $0 \in S$;
\item[(b)] If $x \in S$ then $x+1\in S$ and $x-1\in S$; and
\item[(c)] If $x\in S$ and $x\not\in\{0,1\}$, then $1/(x(x-1))\in S$.
\end{enumerate}
Must $S$ contain all rational numbers?

\item[A--5]
Is there a finite abelian group $G$ such that the product of the
orders of all its elements is $2^{2009}$?

\item[A--6]
Let $f:[0,1]^2 \to \mathbb{R}$ be a continuous function on the closed unit
square such that $\frac{\partial f}{\partial x}$ and $\frac{\partial f}{\partial y}$ exist
and are continuous on the interior $(0,1)^2$. Let $a = \int_0^1 f(0,y)\,dy$,
$b = \int_0^1 f(1,y)\,dy$, $c = \int_0^1 f(x,0)\,dx$, $d = \int_0^1 f(x,1)\,dx$.
Prove or disprove: There must be a point $(x_0,y_0)$ in $(0,1)^2$ such that
\[
\frac{\partial f}{\partial x} (x_0,y_0) = b - a
\quad \mbox{and} \quad
\frac{\partial f}{\partial y} (x_0,y_0) = d - c.
\]

\item[B--1]
Show that every positive rational number can be written as a quotient of products of factorials
of (not necessarily distinct) primes. For example,
\[
\frac{10}{9} = \frac{2!\cdot 5!}{3!\cdot 3! \cdot 3!}.
\]
\,

\item[B--2]
A game involves jumping to the right on the real number line. If $a$ and $b$ are real numbers
and $b > a$, the cost of jumping from $a$ to $b$ is $b^3-ab^2$. For what real numbers
$c$ can one travel from $0$ to $1$ in a finite number of jumps with total cost exactly $c$?

\item[B--3]
Call a subset $S$ of $\{1, 2, \dots, n\}$ \emph{mediocre} if it has the following property:
Whenever $a$ and $b$ are elements of $S$ whose average is an integer, that average is also
an element of $S$. Let $A(n)$ be the number of mediocre subests of $\{1,2,\dots,n\}$.
[For instance, every subset of $\{1,2,3\}$ except $\{1,3\}$ is mediocre, so $A(3) =7$.]
Find all positive integers $n$ such that $A(n+2) - 2A(n+1) + A(n) = 1$.

\item[B--4]
Say that a polynomial with real coefficients in two variables, $x,y$, is \emph{balanced} if
the average value of the polynomial on each circle centered at the origin is $0$.
The balanced polynomials of degree at most $2009$ form a vector space $V$ over $\mathbb{R}$.
Find the dimension of $V$.

\item[B--5]
Let $f: (1, \infty) \to \mathbb{R}$ be a differentiable function such that
\[
 f'(x) = \frac{x^2 - (f(x))^2}{x^2((f(x))^2 + 1)}
\qquad \mbox{for all $x>1$.}
\]
Prove that $\lim_{x \to \infty} f(x) = \infty$.

\item[B--6]
Prove that for every positive integer $n$, there is a sequence of integers
$a_0, a_1, \dots, a_{2009}$ with $a_0 = 0$ and $a_{2009} = n$ such that each term
after $a_0$ is either an earlier term plus $2^k$ for some nonnegative integer $k$,
or of the form $b\,\mathrm{mod}\,c$ for some earlier positive terms $b$ and $c$.
[Here $b\,\mathrm{mod}\,c$ denotes the remainder when $b$ is divided by $c$,
so $0 \leq (b\,\mathrm{mod}\,c) < c$.]

\end{itemize}

\end{document}
