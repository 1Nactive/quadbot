\documentclass[amssymb,twocolumn,pra,10pt,aps]{revtex4-1}
\usepackage{mathptmx,amsmath}

\begin{document}
\title{The 62nd William Lowell Putnam Mathematical Competition \\
    Saturday, December 1, 2001}
\maketitle

\begin{itemize}
\item[A--1]
Consider a set $S$ and a binary operation $*$, i.e., for each $a,b\in S$,
$a*b\in S$.  Assume $(a*b)*a=b$ for all $a,b\in S$.  Prove that
$a*(b*a)=b$ for all $a,b\in S$.

\item[A--2]
You have coins $C_1,C_2,\ldots,C_n$.  For each $k$, $C_k$ is biased so
that, when tossed, it has probability $1/(2k+1)$ of falling heads.
If the $n$ coins are tossed, what is the probability that the number of
heads is odd?  Express the answer as a rational function of $n$.

\item[A--3]
For each integer $m$, consider the polynomial
\[P_m(x)=x^4-(2m+4)x^2+(m-2)^2.\] For what values of $m$ is $P_m(x)$
the product of two non-constant polynomials with integer coefficients?

\item[A--4]
Triangle $ABC$ has an area 1.  Points $E,F,G$ lie, respectively,
on sides $BC$, $CA$, $AB$ such that $AE$ bisects $BF$ at point $R$,
$BF$ bisects $CG$ at point $S$, and $CG$ bisects $AE$ at point $T$.
Find the area of the triangle $RST$.

\item[A--5]
Prove that there are unique positive integers $a$, $n$ such that
$a^{n+1}-(a+1)^n=2001$.

\item[A--6]
Can an arc of a parabola inside a circle of radius 1 have a length
greater than 4?

\item[B--1]
Let $n$ be an even positive integer.  Write the numbers
$1,2,\ldots,n^2$ in the squares of an $n\times n$ grid so that the
$k$-th row, from left to right, is
\[(k-1)n+1,(k-1)n+2,\ldots, (k-1)n+n.\]
Color the squares of the grid so that half of the squares in each
row and in each column are red and the other half are black (a
checkerboard coloring is one possibility).  Prove that for each
coloring, the sum of the numbers on the red squares is equal to
the sum of the numbers on the black squares.

\item[B--2]
Find all pairs of real numbers $(x,y)$ satisfying the system
of equations
\begin{align*}
 \frac{1}{x} + \frac{1}{2y} &= (x^2+3y^2)(3x^2+y^2) \\
   \frac{1}{x} - \frac{1}{2y} &= 2(y^4-x^4).
\end{align*}

\item[B--3]
For any positive integer $n$, let $\langle n\rangle$ denote
the closest integer to $\sqrt{n}$.  Evaluate
\[\sum_{n=1}^\infty \frac{2^{\langle n\rangle}+2^{-\langle n\rangle}}
                         {2^n}.\]

\item[B--4]
Let $S$ denote the set of rational numbers different from
$\{-1,0,1\}$.  Define $f:S\rightarrow S$ by $f(x)=x-1/x$.  Prove
or disprove that
\[\bigcap_{n=1}^\infty f^{(n)}(S) = \emptyset,\]
where $f^{(n)}$ denotes $f$ composed with itself $n$ times.

\item[B--5]
Let $a$ and $b$ be real numbers in the interval $(0,1/2)$, and
let $g$ be a continuous real-valued function such that
$g(g(x))= ag(x)+bx$ for all real $x$.  Prove that
$g(x)=cx$ for some constant $c$.

\item[B--6]
Assume that $(a_n)_{n\geq 1}$ is an increasing sequence of
positive real numbers such that
$\lim a_n/n=0$.  Must there exist infinitely many positive integers
$n$ such that $a_{n-i}+a_{n+i}<2a_n$ for $i=1,2,\ldots,n-1$?

\end{itemize}
\end{document}
