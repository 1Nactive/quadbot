\documentclass[amssymb,twocolumn,pra,10pt,aps]{revtex4-1}
\usepackage{mathptmx,amsmath}

\begin{document}
\title{The 66th William Lowell Putnam Mathematical Competition \\
    Saturday, December 3, 2005}
\maketitle

\begin{itemize}

\item[A--1]
Show that every positive integer is a sum of one or more numbers of the
form $2^r 3^s$, where $r$ and $s$ are nonnegative integers and no
summand divides another.
(For example, 23 = 9 + 8 + 6.)

\item[A--2]
Let $\mathbf{S} = \{(a,b) | a = 1, 2, \dots,n, b = 1,2,3\}$.
A \emph{rook tour} of $\mathbf{S}$ is a polygonal path made up of line
segments connecting points $p_1, p_2, \dots, p_{3n}$ in sequence such that
\begin{enumerate}
\item[(i)] $p_i \in \mathbf{S}$,
\item[(ii)] $p_i$ and $p_{i+1}$ are a unit distance apart, for
$1 \leq i <3n$,
\item[(iii)] for each $p \in \mathbf{S}$ there is a unique $i$ such that
$p_i = p$. How many rook tours are there that begin at $(1,1)$
and end at $(n,1)$?
\end{enumerate}
(An example of such a rook tour for $n=5$ was depicted in the original.)

\item[A--3]
Let $p(z)$ be a polynomial of degree $n$, all of whose zeros have absolute
value 1 in the complex  plane. Put $g(z) = p(z)/z^{n/2}$. Show that all zeros
of $g'(z) = 0$ have absolute value 1.

\item[A--4]
Let $H$ be an $n \times n$ matrix all of whose entries are $\pm 1$ and
whose rows are mutually orthogonal. Suppose $H$ has an $a \times b$ submatrix
whose entries are all $1$. Show that $ab \leq n$.

\item[A--5]
Evaluate
\[
\int_0^1 \frac{\ln(x+1)}{x^2+1}\,dx.
\]

\item[A--6]
Let $n$ be given, $n \geq 4$, and suppose that $P_1, P_2, \dots, P_n$
are $n$ randomly, independently and uniformly, chosen points on a circle.
Consider the convex $n$-gon whose vertices are $P_i$. What is the
probability that at least one of the vertex angles of this polygon is
acute?

\item[B--1]
Find a nonzero polynomial $P(x,y)$ such that $P(\lfloor a \rfloor,
\lfloor 2a \rfloor) = 0$ for all real numbers $a$.
(Note: $\lfloor \nu \rfloor$ is the greatest integer less than
or equal to $\nu$.)

\item[B--2]
Find all positive integers $n, k_1, \dots, k_n$ such that
$k_1 + \cdots + k_n = 5n-4$ and
\[
\frac{1}{k_1} + \cdots + \frac{1}{k_n} = 1.
\]

\item[B--3]
Find all differentiable functions $f: (0, \infty) \to (0, \infty)$ for which
there is a positive real number $a$ such that
\[
f' \left( \frac{a}{x} \right) = \frac{x}{f(x)}
\]
for all $x > 0$.

\item[B--4]
For positive integers $m$ and $n$, let $f(m,n)$ denote the number of
$n$-tuples $(x_1,x_2,\dots,x_n)$ of integers such that
$|x_1| + |x_2| + \cdots + |x_n| \leq m$.
Show that $f(m,n) = f(n,m)$.

\item[B--5]
Let $P(x_1,\dots,x_n)$ denote a polynomial with real coefficients in the
variables $x_1, \dots, x_n$, and suppose that
\[
\left( \frac{\partial^2}{\partial x_1^2} + \cdots + \frac{\partial^2}{\partial
x_n^2}\right) P(x_1, \dots,x_n) = 0 \quad \mbox{(identically)}
\]
and that
\[
x_1^2 + \cdots + x_n^2 \mbox{ divides } P(x_1, \dots, x_n).
\]
Show that $P=0$ identically.

\item[B--6]
Let $S_n$ denote the set of all permutations of the numbers $1,2,\dots,n$.
For $\pi \in S_n$, let $\sigma(\pi) = 1$ if $\pi$ is an even permutation
and $\sigma(\pi) = -1$ if $\pi$ is an odd permutation.
Also, let $\nu(\pi)$ denote the number of fixed points of $\pi$.
Show that
\[
\sum_{\pi \in S_n} \frac{\sigma(\pi)}{\nu(\pi) + 1} = (-1)^{n+1}
\frac{n}{n+1}.
\]
\end{itemize}
\end{document}
