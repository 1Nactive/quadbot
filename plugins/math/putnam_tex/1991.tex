\documentclass[amssymb,twocolumn,pra,10pt,aps]{revtex4-1}
\usepackage{mathptmx,amsmath}

\begin{document}
\title{The 52nd William Lowell Putnam Mathematical Competition \\
    Saturday, December 7, 1991}
\maketitle

\newcommand{\bA}{{\mathbf{A}}}
\newcommand{\bB}{{\mathbf{B}}}
\newcommand{\Z}{{\mathbb{Z}}}

\begin{itemize}
\item[A--1]
A $2 \times 3$ rectangle has vertices as $(0, 0), (2,0), (0,3),$ and $(2,
3)$. It rotates $90^\circ$ clockwise about the point $(2, 0)$. It then
rotates $90^\circ$ clockwise about the point $(5, 0)$, then $90^\circ$
clockwise about the point $(7, 0)$, and finally, $90^\circ$ clockwise
about the point $(10, 0)$. (The side originally on the $x$-axis is now
back on the $x$-axis.) Find the area of the region above the $x$-axis and
below the curve traced out by the point whose initial position is (1,1).

\item[A--2]
Let $\bA$ and $\bB$ be different $n \times n$ matrices with real entries.
If $\bA^3 = \bB^3$ and $\bA^2 \bB = \bB^2 \bA$, can $\bA^2 + \bB^2$ be
invertible?

\item[A--3]
Find all real polynomials $p(x)$ of degree $n \geq 2$ for which there
exist real numbers $r_1 < r_2 < \cdots < r_n$ such that
\begin{enumerate}
    \item $p(r_i) = 0, \qquad i = 1, 2, \dots, n,$ and
    \item $p' \left( \frac{r_i + r_{i+1}}{2} \right) = 0 \qquad i = 1, 2,
    \dots, n-1,$
\end{enumerate}
where $p'(x)$ denotes the derivative of $p(x)$.

\item[A--4]
Does there exist an infinite sequence of closed discs $D_1, D_2, D_3,
\dots$ in the plane, with centers $c_1, c_2, c_3, \dots$, respectively,
such that
\begin{enumerate}
    \item the $c_i$ have no limit point in the finite plane,
    \item the sum of the areas of the $D_i$ is finite, and
    \item every line in the plane intersects at least one of the $D_i$?
\end{enumerate}

\item[A--5]
Find the maximum value of
\[
\int_0^y \sqrt{x^4 + (y-y^2)^2}\,dx
\]
for $0 \leq y \leq 1$.

\item[A--6]
Let $A(n)$ denote the number of sums of positive integers
\[
a_1 + a_2 + \cdots + a_r
\]
which add up to $n$ with
\begin{gather*}
a_1 > a_2 + a_3, a_2 > a_3 + a_4, \ldots, \\
a_{r-2} > a_{r-1} + a_r, a_{r-1}  > a_r.
\end{gather*}
Let $B(n)$ denote the number of $b_1 + b_2 + \cdots + b_s$ which add up
to $n$, with
\begin{enumerate}
    \item $b_1 \geq b_2 \geq \dots \geq b_s,$
    \item each $b_i$ is in the sequence $1, 2, 4, \dots, g_j, \dots$
    defined by $g_1 = 1$, $g_2 = 2$, and $g_j = g_{j-1} + g_{j-2} + 1,$ and
    \item if $b_1 = g_k$ then every element in $\{1, 2, 4, \dots, g_k\}$
    appears at least once as a $b_i$.
\end{enumerate}
Prove that $A(n) = B(n)$ for each $n \geq 1$.

(For example, $A(7) = 5$ because the relevant sums are $7, 6+1, 5+2, 4+3,
4+2+1,$ and $B(7) = 5$ because the relevant sums are $4+2+1, 2+2+2+1,
2+2+1+1+1, 2+1+1+1+1+1, 1+1+1+1+1+1+1.$)

\item[B--1]
For each integer $n \geq 0$, let $S(n) = n - m^2$, where $m$ is the
greatest integer with $m^2 \leq n$. Define a sequence
$(a_k)_{k=0}^\infty$ by $a_0 = A$ and $a_{k+1} = a_k + S(a_k)$ for $k
\geq 0$. For what positive integers $A$ is this sequence eventually constant?

\item[B--2]
Suppose $f$ and $g$ are non-constant, differentiable, real-valued
functions defined on $(-\infty, \infty)$. Furthermore, suppose that for
each pair of real numbers $x$ and $y$,
\begin{align*}
f(x+y) &= f(x)f(y) - g(x)g(y), \\
g(x+y) &= f(x)g(y) + g(x)f(y).
\end{align*}
If $f'(0) = 0$, prove that $(f(x))^2 + (g(x))^2 = 1$ for all $x$.

\item[B--3]
Does there exist a real number $L$ such that, if $m$ and $n$ are integers
greater than $L$, then an $m \times n$ rectangle may be expressed as a
union of $4 \times 6$ and $5 \times 7$ rectangles, any two of which
intersect at most along their boundaries?

\item[B--4]
Suppose $p$ is an odd prime. Prove that
\[
\sum_{j=0}^p \binom{p}{j} \binom{p+j}{j} \equiv 2^p + 1\pmod{p^2}.
\]

\item[B--5]
Let $p$ be an odd prime and let $\Z_p$ denote (the field of) integers
modulo $p$. How many elements are in the set
\[
\{x^2: x \in \Z_p\} \cap \{y^2 + 1 : y \in \Z_p\}?
\]

\item[B--6]
Let $a$ and $b$ be positive numbers. Find the largest number $c$, in
terms of $a$ and $b$, such that
\[
a^x b^{1-x} \leq a \frac{\sinh ux}{\sinh u} + b \frac{\sinh u(1-x)}{\sinh u}
\]
for all $u$ with $0 < |u| \leq c$ and for all $x$, $0 < x < 1$. (Note:
$\sinh u = (e^u - e^{-u})/2$.)
\end{itemize}
\end{document}
