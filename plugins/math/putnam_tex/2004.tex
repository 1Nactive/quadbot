\documentclass[amssymb,twocolumn,pra,10pt,aps]{revtex4-1}
\usepackage{mathptmx,amsmath}

\begin{document}
\title{The 65th William Lowell Putnam Mathematical Competition \\
    Saturday, December 4, 2004}
\maketitle

\begin{itemize}

\item[A--1]
Basketball star Shanille O'Keal's team statistician
keeps track of the number, $S(N)$, of successful free throws she has made
in her first $N$ attempts of the season.
Early in the season, $S(N)$  was less than 80\% of  $N$,
but by the end of the season, $S(N)$ was more than 80\% of $N$.
Was there necessarily a moment in between when $S(N)$ was exactly 80\% of
$N$?

\item[A--2]
For $i = 1,2$ let $T_i$ be a triangle with side lengths $a_i, b_i, c_i$,
and area $A_i$.  Suppose that $a_1 \le a_2,  b_1 \le b_2,  c_1 \le
c_2$,
and that $T_2$ is an acute triangle. Does it follow that $A_1
\le A_2$?

\item[A--3]
Define a sequence $\{ u_n \}_{n=0}^\infty$
by  $u_0 = u_1 = u_2 = 1$, and thereafter by
the
condition that
\[
\det\begin{pmatrix}
u_n &   u_{n+1}\\
u_{n+2} & u_{n+3}
\end{pmatrix}
= n!
\]
for all $n \ge 0$. Show that $u_n$ is an integer for all $n$.
(By convention, $0! = 1$.)

\item[A--4]
Show that for any positive integer $n$, there is an integer $N$ such that
the product $x_1 x_2 \cdots x_n$ can be expressed identically in the form
\[
x_1 x_2 \cdots x_n =
\sum_{i=1}^N  c_i
( a_{i1} x_1 + a_{i2} x_2 + \cdots + a_{in} x_n )^n
\]
where the $c_i$ are rational numbers and each $a_{ij}$ is one of the
numbers $-1, 0, 1$.

\item[A--5]
An $m \times n$ checkerboard is colored randomly: each square is
independently
assigned red or black with probability $1/2$. We say that two squares,
$p$ and $q$,  are in the same connected monochromatic component if there is
a sequence of squares, all of the same color, starting at $p$ and ending
at $q$, in which successive squares in the sequence share a common side.
Show that the expected number of connected monochromatic regions is
greater than $m n / 8$.

\item[A--6]
Suppose that $f(x,y)$ is a continuous real-valued function on the unit
square $0 \le x \le 1, 0 \le y \le 1$.  Show that
\begin{align*}
& \int_0^1 \left( \int_0^1  f(x,y) dx \right)^2 dy +
 \int_0^1 \left( \int_0^1  f(x,y) dy \right)^2 dx \\
&\leq
\left( \int_0^1 \int_0^1  f(x,y) dx\, dy \right)^2 +
\int_0^1 \int_0^1 \left( f(x,y) \right)^2 dx\,dy.
\end{align*}

\item[B--1]
Let $P(x) = c_n x^n + c_{n-1} x^{n-1} + \cdots + c_0$ be a polynomial with
integer coefficients. Suppose that $r$ is a rational number such that
$P(r) = 0$.  Show that the $n$ numbers
\begin{gather*}
c_n r, \, c_n r^2 + c_{n-1} r, \, c_n r^3 + c_{n-1} r^2 + c_{n-2} r, \\
\dots, \, c_n r^n + c_{n-1} r^{n-1} + \cdots + c_1 r
\end{gather*}
are integers.

\item[B--2]
Let $m$ and $n$ be positive integers.  Show that
\[
\frac{(m+n)!}{(m+n)^{m+n}}
< \frac{m!}{m^m} \frac{n!}{n^n}.
\]

\item[B--3]
Determine all real numbers $a > 0$ for which there exists a nonnegative
continuous function $f(x)$ defined on $[0,a]$ with the property that the
region
\[
R = \{ (x,y) ; 0
\le x \le a, 0 \le y \le
f(x) \}
\]
has perimeter $k$ units and area $k$ square units for some real number $k$.

\item[B--4]
Let $n$ be a positive integer, $n \ge
2$, and put  $\theta = 2 \pi / n$.
Define points $P_k = (k,0)$ in the $xy$-plane, for $k = 1, 2
, \dots, n$.
Let $R_k$ be the map that rotates the plane counterclockwise by the
angle $\theta$ about the point $P_k$.  Let $R$ denote the map obtained
by applying, in order, $R_1$,  then $R_2, \dots$,
then $R_n$.
For an arbitrary point $(x,y)$, find, and simplify, the coordinates
of $R(x,y)$.

\item[B--5]
Evaluate
\[
\lim_{x \to 1^-} \prod_{n=0}^\infty \left(\frac{1 + x^{n+1}}{1 +
x^n}\right)^{x^n}.
\]

\item[B--6]
Let $\mathcal{A}$
be a non-empty set of positive integers, and let $N(x)$ denote
the number of elements of $\mathcal{A}$ not exceeding $x$.
Let $\mathcal{B}$ denote the set
of positive integers $b$ that can be written in the form $b = a - a'$ with
$a \in \mathcal{A}$  and $a' \in  \mathcal{A}$. Let $b_1 < b_2 < \cdots$
be the members of $\mathcal{B}$,
listed in increasing order. Show that if the sequence $b_{i+1} - b_i$ is
unbounded, then
\[
\lim_{x \to\infty}  N(x)/x  = 0.
\]

\end{itemize}
\end{document}
