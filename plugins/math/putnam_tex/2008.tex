\documentclass[amssymb,twocolumn,pra,10pt,aps]{revtex4-1}
\usepackage{mathptmx,amsmath}

\begin{document}
\title{The 69th William Lowell Putnam Mathematical Competition \\
    Saturday, December 6, 2008}
\maketitle

\begin{itemize}

\item[A--1]
Let $f: \mathbb{R}^2 \to \mathbb{R}$ be a function such that $f(x,y) + f(y,z)
+ f(z,x) = 0$ for all real numbers $x$, $y$, and $z$. Prove that there exists
a function $g: \mathbb{R} \to \mathbb{R}$ such that $f(x,y) = g(x) - g(y)$
for all real numbers $x$ and $y$.

\item[A--2]
Alan and Barbara play a game in which they take turns filling entries of
an initially empty $2008 \times 2008$ array. Alan plays first. At each
turn, a player chooses a real number and places it in a vacant entry.
The game ends when all the entries are filled. Alan wins if the
determinant of the resulting matrix is nonzero; Barbara wins if it is zero.
Which player has a winning strategy?

\item[A--3]
Start with a finite sequence $a_1, a_2, \dots, a_n$ of positive integers.
If possible, choose two indices $j < k$ such that $a_j$ does not divide
$a_k$, and replace $a_j$ and $a_k$ by $\mathrm{gcd}(a_j, a_k)$
and $\mathrm{lcm}(a_j, a_k)$, respectively. Prove that if this process is
repeated, it must eventually stop and the final sequence does not depend
on the choices made. (Note: gcd means greatest common divisor and lcm
means least common multiple.)

\item[A--4]
Define $f: \mathbb{R} \to \mathbb{R}$ by
\[
f(x) = \begin{cases} x & \mbox{if $x \leq e$} \\ x f(\ln x) &
\mbox{if $x > e$.} \end{cases}
\]
Does $\sum_{n=1}^\infty \frac{1}{f(n)}$ converge?

\item[A--5]
Let $n \geq 3$ be an integer. Let $f(x)$ and $g(x)$ be polynomials
with real coefficients such that the points
$(f(1), g(1)), (f(2), g(2)), \dots, (f(n), g(n))$
in $\mathbb{R}^2$ are the vertices of a regular $n$-gon in
counterclockwise order. Prove that at least one of $f(x)$
and $g(x)$ has degree greater than or equal to $n-1$.

\item[A--6]
Prove that there exists a constant $c>0$ such that in every
nontrivial finite group $G$ there exists a sequence of length
at most $c \ln |G|$ with the property that each element of $G$
equals the product of some subsequence. (The elements of $G$ in the
sequence are not required to be distinct. A \emph{subsequence}
of a sequence is obtained by selecting some of the terms,
not necessarily consecutive, without reordering them; for
example, $4, 4, 2$ is a subsequence of $2, 4, 6, 4, 2$, but
$2, 2, 4$ is not.)

\item[B--1]
What is the maximum number of rational points that can lie on a circle
in $\mathbb{R}^2$ whose center is not a rational point? (A \emph{rational
point} is a point both of whose coordinates are rational numbers.)

\item[B--2]
Let $F_0(x) = \ln x$. For $n \geq 0$ and $x > 0$, let
$F_{n+1}(x) = \int_0^x F_n(t)\,dt$. Evaluate
\[
\lim_{n \to \infty} \frac{n! F_n(1)}{\ln n}.
\]

\item[B--3]
What is the largest possible radius of a circle contained in a 4-dimensional
hypercube of side length 1?

\item[B--4]
Let $p$ be a prime number. Let $h(x)$ be a polynomial with integer coefficients
such that $h(0), h(1), \dots, h(p^2-1)$ are distinct modulo $p^2$.
Show that $h(0), h(1), \dots, h(p^3-1)$ are distinct modulo $p^3$.

\item[B--5]
Find all continuously differentiable functions $f: \mathbb{R} \to \mathbb{R}$
such that for every rational number $q$, the number $f(q)$ is rational
and has the same denominator as $q$. (The denominator of a rational number
$q$ is the unique positive integer $b$ such that $q = a/b$
for some integer $a$ with $\mathrm{gcd}(a,b) = 1$.)
(Note: gcd means greatest common
divisor.)

\item[B--6]
Let $n$ and $k$ be positive integers. Say that a permutation $\sigma$
of $\{1,2,\dots,n\}$ is \emph{$k$-limited} if $|\sigma(i) - i| \leq k$
for all $i$. Prove that the number of $k$-limited permutations of
$\{1,2,\dots,n\}$ is odd if and only if $n \equiv 0$ or $1$
(mod $2k+1$).

\end{itemize}

\end{document}
